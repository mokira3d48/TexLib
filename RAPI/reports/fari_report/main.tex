\documentclass[12pt, a4paper]{article}


\usepackage{arxiv}

\usepackage[utf8]{inputenc} % allow utf-8 input
\usepackage[french]{babel}
\usepackage[T1]{fontenc}    % use 8-bit T1 fonts
\usepackage{hyperref}       % hyperlinks
\usepackage{url}            % simple URL typesetting
\usepackage{booktabs}       % professional-quality tables
\usepackage{amsfonts}       % blackboard math symbols
\usepackage{nicefrac}       % compact symbols for 1/2, etc.
\usepackage{microtype}      % microtypography
\usepackage{lipsum}
\usepackage{graphicx}
\graphicspath{ {./images/} }

\usepackage{fontsize}
\changefontsize{12}

\setlength{\parindent}{0pt} % Paragraph indentation


\title{Rapport sur le déroulement de l'événement FARI 2025 (Forum Africa pour la Recherche et l'Innovation)}


\author{
 Arnold (Mokira) T. \\
 Ingénieur Machine Learning en formation \\
  \texttt{dr.mokira@uatmgasa.com} \\
  %% examples of more authors
   \And
 Farhanath M. \\
 Ingénieur Data en formation \\
  \texttt{farhanathmoustapha@uatmgasa.com} \\
  \And
 Prince (Obrymec) T. \\
 Développeur front-end \\
  \texttt{obrymecsprinces@uatmgasa.com} \\
  %% \AND
  %% Coauthor \\
  %% Affiliation \\
  %% Address \\
  %% \texttt{email} \\
  %% \And
  %% Coauthor \\
  %% Affiliation \\
  %% Address \\
  %% \texttt{email} \\
  %% \And
  %% Coauthor \\
  %% Affiliation \\
  %% Address \\
  %% \texttt{email} \\
}

\begin{document}
\maketitle
\begin{abstract}
Notre présence à l'événement FARI a été marquée par un intérêt soutenu des visiteurs et des rencontres stratégiques. Dès le premier jour, nous avons reçu la visite du responsable des stands et du Directeur général de la SBEE. Les jours suivants, des visiteurs (dont des stagiaires et le Directeur des Systèmes d'Information) ont montré un intérêt particulier pour nos projets, notamment la station solaire amovible et le système de détection de plaques d'immatriculation. Le Directeur des DSI a exprimé son souhait de collaborer avec le laboratoire GASA-Formation pour des projets concrets. L'organisation sur nos deux stands (administration/présentateurs) a permis un accueil fluide des visiteurs, avec des présentations appréciées et une collecte coordonnée de contacts. La logistique quotidienne (repas, transport) a été assurée efficacement.
\end{abstract}


% keywords can be removed
%\keywords{First keyword \and Second keyword \and More}


\section{Introduction}
Notre participation à l'édition 2025 de FARI (5-8 mai) a constitué une vitrine stratégique pour nos projets technologiques. Le bilan ci-dessous rend compte des visites marquantes, de l'accueil favorable réservé à nos démonstrations, et des opportunités de partenariat émergeantes, notamment avec la SBEE et la Direction des Systèmes d'Information.

\section{En bref}
Notre espace d'exposition comprenait deux stands distincts :

\begin{itemize}
	\item[$\bullet$] Le premier était occupé par les administrateurs (M. Arnaud et les membres du service de renseignement) accompagnés de quelques étudiants de master ;
	\item[$\bullet$] Le second par les présentateurs de projets (étudiants en soutenance et notre équipe d'informaticiens).
\end{itemize}

Lorsqu'un visiteur s'approchait d'un stand ou d'une exposition, les équipes et M. Arnaud l'accueillaient immédiatement pour entamer une présentation. En fin de démonstration, la plupart des visiteurs - satisfaits des explications - laissaient leurs coordonnées (cartes de visite ou numéros de téléphone). Ces informations étaient systématiquement remises à M. Arnaud, sauf si le visiteur choisissait de les partager avec toute l'équipe.

\subsection{Concernant la logistique}

\begin{itemize}
	\item[$\bullet$] Chaque après-midi (entre 14h et 15h), un chauffeur apportait des repas que nous ramenions sur les stands ;
	\item[$\bullet$] La distribution des repas et boissons était assurée à tour de rôle par un membre de notre équipe ou du service de renseignement ;
	\item[$\bullet$] Vers 18h30, chaque groupe rangeait son exposition avant le retour organisé par les chauffeurs vers 19h.
\end{itemize}

\section{Détails}
\label{sec:details}
\paragraph{05 Mai 2025}
Peu après notre installation, nous avons été visités par le responsable des stands et le Directeur général de la SBEE. Ils se sont particulièrement intéressés au station solaire et groupe électrogène statique, en nous questionnant sur son coût de fabrication et les économies d'énergie qu'il permet. Par la suite, de nombreux visiteurs sont venus découvrir nos projets et sont repartis satisfaits.

\paragraph{06 Mai 2025}
Au deuxième jour de l’exposition, nous avons également reçu la visite de plusieurs visiteurs dont la plupart étaient intèressés par la station solaire amovible présenter par les étudiants en licence. Un peu dans l'après-midi, nous avions reçu, certaines personnes venue d'autres pays comme le Gabon qui ce sont aussi intéressés à notre système de détection de plaque d'immatriculation, mais n'ont pas cherché à le tester en profondeur. C'est un peu vers la fin de la journée que nous avions reçu un monsieur qui connaissait un autre monsieur qui travaille à la DSI d'une organisation dont on ne se souvient plus du nom ni de la fonction.

Ensuite, le Professeur Degbo Basile (ancien enseignant à GASA-FORMATION) nous a présenté un projet innovant de suivi véhiculaire. Celui-ci repose sur des puces RFID intégrées aux véhicules, contenant leurs données techniques, et des lecteurs fixes installés aux carrefours. Ces dispositifs détectent automatiquement les véhicules à proximité et transmettent les informations lues (immatriculation, heure, position) vers une base de données centralisée.

Ce système présente des avantages importants : sa précision excellente (plus de $99\%$) dépasse les méthodes classiques comme les boucles au sol. De plus, il est facile à installer et peu coûteux à entretenir. Les données récoltées permettent notamment de créer des cartes de trafic en direct ou d’ajuster les feux tricolores.

Cependant, il a une limitation majeure : tous les véhicules doivent être équipés d’une puce RFID. Dans un contexte comme le nôtre, cela représenterait un coût élevé. De plus, des personnes mal intentionnées pourraient retirer la puce pour éviter d’être repérées.

C’est précisément sur ce point que notre système de reconnaissance de plaques d’immatriculation offre une solution complémentaire, car il fonctionne sans équipement spécifique sur les véhicules.

\paragraph{08 Mai 2025}
Le jeudi, nous avons reçu la visite de Mr. Alex Foadey qui est à la Direction des Systèmes d’Information, qui après qu’on lui ait présenté le projet portant sur les plaques d’immatriculation,  nous a parler de son envie de collaborer avec le laboratoire de GASA-Formation afin de pouvoir réaliser certains projets, par exemple le compteur automatique du nombre de voiture qui sont passées par un pompe-payage et à une frontière.
Mr. Alex Foadey nous a laissé son contact WhatsApp \textbf{+229 01 95 051 155}.


\section{Conclusion}
Notre participation au FARI 2025 a constitué une opportunité stratégique tant pour la valorisation de nos innovations technologiques (station solaire, système de détection de plaques) que pour le développement de partenariats prometteurs. L’intérêt manifesté par des acteurs clés comme la SBEE, notamment à travers la proposition de collaboration concrète de M. Foadey (automatisation du comptage véhiculaire), confirme le potentiel applicatif de nos projets.

Par ailleurs, les échanges avec le Pr. Degbo ont enrichi notre réflexion sur les solutions complémentaires en mobilité urbaine, soulignant comment notre approche de reconnaissance visuelle pallie les limites des systèmes RFID dans des contextes comme le nôtre. Forts de ces retours positifs et contacts établis, nous envisageons désormais des prototypes opérationnels en synergie avec les institutions intéressantes, afin de concrétiser ces avancées dans un cadre partenarial structuré.


%\bibliography{references}  %%% Remove comment to use the external .bib file (using bibtex).
%%% and comment out the ``thebibliography'' section.


\end{document}
