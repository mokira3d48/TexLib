% Aide moi a rediger un rapport pour presenter les resultats de recherche en deep learning suivant le plan de redaction ci-dessous. D'abord, il s'agit de developper un model de deep learning qui permet la lecture automatique des plaques d'immatriculation. Au debut, nous avons implementé le modèle de reconnaissance des numéros en utilisant une approache monolitique qui consistait à entrainer un modèle de detection et reconnaissance d'objet nommé YOLO. Il s'agissait de la version 11 de YOLO de ultralytics. Ce modèle fesait bien l'affaire, il nous permettait a la fois de faire la localisation et la classification du numéro en une seule passe. Mais, le problème avec ce modele est qu'il faux assez d'image de plaque labelisées dans la dataset améliorer sa précision d'identification de numéro. Ce qui reviens très couteux en ressource humaine. De plus, pour labeliser une seule image de plaque, il faut plusieurs minutes a une personne. Enfin le mode de fonctionnement du model YOLO ne permettra pas d'atteindre des précisions extrême car pendant l'entrainement, le modele optimise sur la classification de l'objet detecté alors que les boites de detection qu'il propose ne couvre pas encore le numéro en question. Voici les informations sur les performances de notre encien modele (Monolitique) : Ultralytics 8.3.89 🚀 Python-3.10.12 torch-2.7.0+cpu CPU (Intel Core(TM) i5-1035G1 1.00GHz) Class Images Instances Box(P R mAP50 mAP50-95): 100%|██████████| 360/360 [00:08<00:00, 43.63it/s] all 360 2266 0.95 0.922 0.969 0.692 0 112 134 0.953 0.963 0.984 0.719 1 112 132 0.968 0.92 0.967 0.601 2 234 319 0.987 0.98 0.994 0.707 3 123 140 0.975 0.957 0.985 0.687 4 127 146 0.994 0.966 0.99 0.684 5 106 118 0.966 0.966 0.989 0.705 6 128 153 0.975 0.954 0.981 0.706 7 138 165 0.987 0.947 0.971 0.671 8 123 135 0.943 0.948 0.982 0.731 9 112 134 0.949 0.94 0.99 0.699 A 62 62 0.973 0.887 0.951 0.649 B 129 135 0.987 0.97 0.994 0.715 C 90 96 0.989 1 0.995 0.762 D 59 59 0.982 0.983 0.995 0.778 E 52 56 0.941 0.946 0.958 0.719 F 19 19 1 0.893 0.936 0.66 G 20 20 0.891 0.95 0.983 0.649 H 12 12 1 0.822 0.847 0.539 J 7 7 0.952 1 0.995 0.621 K 17 17 0.925 0.882 0.96 0.711 L 15 15 0.962 0.867 0.983 0.559 M 16 16 0.82 0.854 0.891 0.662 N 12 12 0.75 0.833 0.908 0.645 O 1 1 0.819 1 0.995 0.895 P 13 13 0.962 0.846 0.88 0.597 R 27 27 0.942 0.926 0.971 0.742 S 16 16 1 0.733 0.978 0.704 T 15 15 1 0.894 0.995 0.663 U 22 22 0.913 0.955 0.985 0.71 V 19 19 0.945 0.902 0.99 0.723 W 2 4 1 0.794 0.995 0.698 X 18 18 0.97 0.944 0.988 0.771 Y 15 15 0.877 0.933 0.935 0.665 Z 14 14 0.992 1 0.995 0.772 Speed: 0.1ms preprocess, 19.6ms inference, 0.0ms loss, 0.5ms postprocess per image Results saved to runs/detect/val2 Class indices with average precision: [ 0 1 2 3 4 5 6 7 8 9 10 11 12 13 14 15 16 17 19 20 21 22 23 24 25 27 28 29 30 31 32 33 34 35] Mean average precision at IoU=0.50: 0.9687206227602649 Mean average precision at IoU=0.75: 0.8545967599286844 Mean precision: 0.9496859891399704 Mean recall: 0.9222732729430398 Mean F1: 0.9357789170023392 Maintenant, nous avons choisi de diviser pour mieux raigner. Nous avons choisi une architecture hybride dans laquelle mon y trouve un model de localisation des numéros (character detection) et un modele de classification de ces numéros une fois roigné de l'image de plaque de depart. Avec ce modèle, l'annotation des images de plaque est devenu plus facile, car au lieu d'encadrer et de dicerner les caractères à la fois, cherche juste a encadrer un caractère, ce reduit le temps de labelisation de 1/4. De plus, par expérience, nous avons remarque le modele YOLO performe mieux lorsqu'il a un seul objet à detecter que lorsqu'il y a plusieurs objet a detecter. La classification de numéro est devenu plus facile, car il faut juste glisser-deposer une image dans le dossier correcpondant a sa classe. Cela nous permet d'avoir assez d'exemple de donnees afin d'ameliorer la precision du modele de classification de numeros. Le seul point negatif que nous avons avec l'approche hybride est l'augmentation du temps de calcul. Voici les informations de notre modele actuel (hybride) : ============================================================ MODEL EVALUATION RESULTS ============================================================ Total images processed: 2266 Total inference time: 91.0610 seconds Average inference time per image: 0.0402 seconds Accuracy: 0.9585 Precision: 0.9653 Recall: 0.9585 CLASSIFICATION REPORT: ------------------------------------------------------------ precision recall f1-score support 0 0.93 0.93 0.93 134 1 1.00 0.92 0.96 132 2 0.99 0.98 0.98 319 3 0.95 0.98 0.96 140 4 0.99 0.99 0.99 146 5 0.96 0.98 0.97 118 6 0.96 0.99 0.97 153 7 0.95 0.98 0.96 165 8 0.92 0.98 0.95 135 9 0.98 0.97 0.97 134 A 0.97 0.98 0.98 62 B 0.99 0.93 0.96 135 C 0.97 0.99 0.98 96 D 0.96 0.83 0.89 59 E 0.98 0.98 0.98 56 F 1.00 0.89 0.94 19 G 0.86 0.90 0.88 20 H 0.92 1.00 0.96 12 I 0.00 0.00 0.00 0 J 0.67 0.86 0.75 7 K 1.00 0.94 0.97 17 L 1.00 0.80 0.89 15 M 1.00 0.94 0.97 16 N 0.92 0.92 0.92 12 O 0.17 1.00 0.29 1 P 0.75 0.92 0.83 13 Q 0.00 0.00 0.00 0 R 1.00 0.89 0.94 27 S 1.00 0.81 0.90 16 T 1.00 0.93 0.97 15 U 0.95 0.91 0.93 22 V 1.00 0.89 0.94 19 W 0.80 1.00 0.89 4 X 1.00 0.94 0.97 18 Y 0.82 0.93 0.88 15 Z 0.93 0.93 0.93 14 accuracy 0.96 2266 macro avg 0.87 0.88 0.87 2266 weighted avg 0.97 0.96 0.96 2266 Classification Metrics by Confidence Interval ================================================================================ Interval Accuracy Precision Recall F1-Score -------------------------------------------------------------------------------- [0.000 - 0.020[ 0.000 0.000 0.000 0.000 [0.020 - 0.040[ 0.000 0.000 0.000 0.000 [0.040 - 0.060[ 0.000 0.000 0.000 0.000 [0.060 - 0.080[ 0.000 0.000 0.000 0.000 [0.080 - 0.100[ 0.000 0.000 0.000 0.000 [0.100 - 0.120[ 0.000 0.000 0.000 0.000 [0.120 - 0.140[ 0.000 0.000 0.000 0.000 [0.140 - 0.160[ 0.000 0.000 0.000 0.000 [0.160 - 0.180[ 0.000 0.000 0.000 0.000 [0.180 - 0.200[ 0.000 0.000 0.000 0.000 [0.200 - 0.220[ 1.000 1.000 1.000 1.000 [0.220 - 0.240[ 0.250 0.250 0.250 0.250 [0.240 - 0.260[ 0.000 0.000 0.000 0.000 [0.260 - 0.280[ 0.000 0.000 0.000 0.000 [0.280 - 0.300[ 0.000 0.000 0.000 0.000 [0.300 - 0.320[ 0.000 0.000 0.000 0.000 [0.320 - 0.340[ 0.333 0.667 0.333 0.444 [0.340 - 0.360[ 1.000 1.000 1.000 1.000 [0.360 - 0.380[ 0.200 0.200 0.200 0.200 [0.380 - 0.400[ 0.500 0.500 0.500 0.500 [0.400 - 0.420[ 0.000 0.000 0.000 0.000 [0.420 - 0.440[ 0.400 0.400 0.400 0.400 [0.440 - 0.460[ 0.000 0.000 0.000 0.000 [0.460 - 0.480[ 0.000 0.000 0.000 0.000 [0.480 - 0.500[ 0.000 0.000 0.000 0.000 [0.500 - 0.520[ 0.500 0.500 0.500 0.500 [0.520 - 0.540[ 0.250 0.500 0.250 0.333 [0.540 - 0.560[ 0.333 0.500 0.333 0.389 [0.560 - 0.580[ 0.857 1.000 0.857 0.914 [0.580 - 0.600[ 0.250 0.250 0.250 0.250 [0.600 - 0.620[ 0.571 0.571 0.571 0.571 [0.620 - 0.640[ 0.750 0.750 0.750 0.750 [0.640 - 0.660[ 0.857 0.857 0.857 0.857 [0.660 - 0.680[ 0.500 0.375 0.500 0.417 [0.680 - 0.700[ 0.286 0.429 0.286 0.333 [0.700 - 0.720[ 0.800 1.000 0.800 0.867 [0.720 - 0.740[ 0.571 0.500 0.571 0.524 [0.740 - 0.760[ 0.333 0.667 0.333 0.444 [0.760 - 0.780[ 0.727 0.800 0.727 0.726 [0.780 - 0.800[ 0.500 0.500 0.500 0.500 [0.800 - 0.820[ 0.333 0.333 0.333 0.333 [0.820 - 0.840[ 0.636 0.727 0.636 0.673 [0.840 - 0.860[ 0.778 0.722 0.778 0.741 [0.860 - 0.880[ 0.875 0.812 0.875 0.833 [0.880 - 0.900[ 0.692 0.885 0.692 0.731 [0.900 - 0.920[ 0.857 0.857 0.857 0.857 [0.920 - 0.940[ 0.667 0.833 0.667 0.700 [0.940 - 0.960[ 0.886 0.952 0.886 0.909 [0.960 - 0.980[ 0.878 0.918 0.878 0.883 [0.980 - 1.000[ 0.997 0.997 0.997 0.997 Exactly 1.000 1.000 1.000 1.000 1.000 CONFIDENCE STATISTICS (threshold: 0.92): ------------------------------------------------------------ Confident predictions: 2097/2266 (92.54%) Average confidence: 0.9723 Min confidence: 0.1966 Max confidence: 1.0000 Accuracy on confident predictions: 0.9909 Au vue de ces informations ci-dessous redige moi un bref rapport presentant l'essentielle à savoir sur tout ce qui a ete fait juste que là. N'oublis pas de me rediger un petit resumé d'un court paragraphe.

%%%%%%%%%%%%%%%%%%%%%%%%%%%%%%%%%%%%%%%%%%%%%%%%%%%%%%%%%%%%%%%%%%%%%%%%%%%%%%%%
%2345678901234567890123456789012345678901234567890123456789012345678901234567890
%        1         2         3         4         5         6         7         8

\documentclass[letterpaper, 10 pt, conference]{ieeeconf}  % Comment this line out
                                                          % if you need a4paper
%\documentclass[a4paper, 10pt, conference]{ieeeconf}      % Use this line for a4
                                                          % paper

\IEEEoverridecommandlockouts                              % This command is only
                                                          % needed if you want to
                                                          % use the \thanks command
\overrideIEEEmargins
% See the \addtolength command later in the file to balance the column lengths
% on the last page of the document

\usepackage[utf8]{inputenc}
\usepackage[T1]{fontenc}
\usepackage{amsmath}
\usepackage{graphicx}

\title{\LARGE \bf
Rapport (2025.10.27) Reconnaissance Automatique de Plaques d'Immatriculation
}

\author{Équipe de Recherche en Intelligence Artificielle% <-this % stops a space
\thanks{*Ce travail a été réalisé par le département de recherche en IA}% <-this % stops a space
}

% \author{Arnold T.$^{1}$, Prince T.$^{2}$, Farhanath M.$^{3}$% <-this % stops a space
% \thanks{*Ce travail a été soutenu par le département de recherche en intelligence artificielle et vision par ordinateur}% <-this % stops a space
% \thanks{$^{1}$Arnold T. est spécialiste en architectures de deep learning et optimisation de modèles}%
% \thanks{$^{2}$Prince T. est expert en traitement d'images et annotation de données}%
% \thanks{$^{3}$Farhanath M. est responsable de l'évaluation des performances et métriques}%
% }



\begin{document}

\maketitle
\thispagestyle{empty}
\pagestyle{empty}

%%%%%%%%%%%%%%%%%%%%%%%%%%%%%%%%%%%%%%%%%%%%%%%%%%%%%%%%%%%%%%%%%%%%%%%%%%%%%%%%
\begin{abstract}

Ce document présente l'évolution d'un système de reconnaissance automatique de plaques d'immatriculation utilisant le deep learning. Notre recherche compare une approche monolithique basée sur YOLOv11 avec une architecture hybride séparant la détection et la classification des caractères. Les résultats démontrent que l'approche hybride améliore significativement les performances tout en réduisant les coûts d'annotation de 75\%. L'exactitude atteint 95.85\% avec des perspectives d'optimisation vers l'excellence.

\end{abstract}

%%%%%%%%%%%%%%%%%%%%%%%%%%%%%%%%%%%%%%%%%%%%%%%%%%%%%%%%%%%%%%%%%%%%%%%%%%%%%%%%
% \section{INTRODUCTION}

% La reconnaissance automatique de plaques d'immatriculation (ALPR - Automatic License Plate Recognition) représente un défi majeur en vision par ordinateur, avec des applications critiques dans la sécurité, la gestion du trafic et les systèmes de péage. Les principales difficultés incluent la variabilité des conditions d'éclairage, les angles de vue, les résolutions d'image et la diversité des formats de plaques.

% Les approches traditionnelles souffrent de limitations en termes de robustesse et de précision. Notre recherche explore l'évolution des architectures de deep learning, comparant une approche monolithique avec une architecture hybride plus sophistiquée, avec pour objectif d'optimiser le compromis entre performance, coût de développement et facilité de déploiement.

\section{LE PASSÉ : APPROCHE MONOLITHIQUE}

\subsection{Architecture et Caractéristiques}

Notre implémentation initiale utilisait un modèle YOLOv11 (You Only Look Once version 11) monolithique, caractérisé par :

\begin{itemize}
\item \textbf{Architecture unifiée} : Détection et classification simultanées en une seule passe
\item \textbf{Simplicité d'implémentation} : Pipeline unique pour l'ensemble du processus
\item \textbf{Traitement en temps réel} : Optimisé pour des applications nécessitant une faible latence
\end{itemize}

L'approche monolithique présentait cependant des limitations fondamentales :

\begin{equation}
\text{Temps d'annotation} = n \times t_{\text{caractère}} \times c_{\text{complexité}}
\end{equation}

où $n$ représente le nombre d'images, $t_{\text{caractère}}$ le temps par caractère, et $c_{\text{complexité}}$ la complexité de l'annotation simultanée.

\subsection{Performances et Limitations}

Les résultats obtenus avec l'architecture monolithique révélaient des contraintes significatives :

\begin{table}[h]
\caption{Performances du Modèle Monolithique YOLOv11}
\label{table_monolithique}
\begin{center}
\begin{tabular}{|c|c|}
\hline
\textbf{Métrique} & \textbf{Valeur} \\
\hline
mAP50 & 96.87\% \\
\hline
mAP50-95 & 69.20\% \\
\hline
Précision & 94.97\% \\
\hline
Rappel & 92.23\% \\
\hline
Score F1 & 93.58\% \\
\hline
Temps d'inférence & 19.6ms \\
\hline
\end{tabular}
\end{center}
\end{table}

L'analyse critique de ces résultats montre que bien que le mAP50 soit acceptable, le mAP50-95 modeste indique une sensibilité aux variations de positionnement et de taille des caractères. La nature conflictuelle de l'optimisation simultanée de la détection et de la classification limitait les performances maximales atteignables.

\section{LE PRÉSENT : ARCHITECTURE HYBRIDE}

\subsection{Conception du Nouveau Système}

Face aux limitations du modèle monolithique, nous avons développé une architecture hybride décomposant le processus en deux modules spécialisés :

\begin{itemize}
\item \textbf{Module de détection} : YOLO optimisé pour la localisation précise des caractères
\item \textbf{Module de classification} : Réseau neuronal dédié à l'identification des caractères rognés
\item \textbf{Pipeline séquentiel} : Traitement en deux étapes distinctes mais intégrées
\end{itemize}

L'architecture hybride offre des avantages significatifs :

\begin{equation}
\text{Gain d'annotation} = \frac{t_{\text{monolithique}} - t_{\text{hybride}}}{t_{\text{monolithique}}} = 75\%
\end{equation}

\subsection{Performances et Analyse}

Les résultats de l'architecture hybride démontrent une amélioration substantielle :

\begin{table}[h]
\caption{Performances du Modèle Hybride}
\label{table_hybride}
\begin{center}
\begin{tabular}{|c|c|}
\hline
\textbf{Métrique} & \textbf{Valeur} \\
\hline
mAP50 & 99.47\% \\
\hline
mAP50-95 & 74.01\% \\
\hline
Précision & 96.53\% \\
\hline
Rappel & 95.85\% \\
\hline
Temps d'inférence & 0.0402s \\
\hline
Prédictions confiantes & 92.54\% \\
\hline
Précision des prédictions confiantes (> 92 \%) & 99.09\% \\
\hline
\end{tabular}
\end{center}
\end{table}

L'analyse par classe révèle des performances exceptionnelles pour certains caractères :

\begin{equation}
\text{F1-score}_{\text{meilleur}} = 0.99 \quad \text{(caractères 2, 4, 9, C, E, K, M)}
\end{equation}

tandis que certaines classes nécessitent une attention particulière :

\begin{equation}
\text{F1-score}_{\text{faible}} < 0.90 \quad \text{(lettres O, J, P, D)}
\end{equation}

\subsection{Avantages Constatés}

L'architecture hybride présente plusieurs avantages décisifs :

\begin{itemize}
\item \textbf{Spécialisation} : Chaque module optimisé pour sa tâche spécifique
\item \textbf{Réduction des coûts} : Annotation 4 fois plus rapide
\item \textbf{Flexibilité} : Amélioration indépendante des modules
\item \textbf{Robustesse} : Meilleure gestion des variations et du bruit
\end{itemize}

Le principal défi reste l'augmentation du temps de calcul global, compensé par les gains en précision et facilité de maintenance.

\section{LE FUTURE : PERSPECTIVES D'AMÉLIORATION}

Notre feuille de route stratégique vise l'excellence avec des objectifs quantifiés précis :

\subsection{Objectifs Principaux}

\begin{equation}
\text{F1-score}_{\text{cible}} = 99\% \quad \text{et} \quad \mathcal{L}_{\text{CrossEntropy}} = 1 \times 10^{-3}
\end{equation}

où $\mathcal{L}_{\text{CrossEntropy}}$ représente la fonction de perte d'entropie croisée.

\subsection{Stratégie d'Optimisation}

Pour atteindre ces objectifs ambitieux, nous mettons en œuvre une stratégie multi-facettes :

\begin{itemize}
\item \textbf{Augmentation des données} : Collecte massive d'échantillons supplémentaires
\item \textbf{Data augmentation avancée} : Techniques de génération synthétique
\item \textbf{Optimisation d'architecture} : Fine-tuning des modèles actuels
\item \textbf{Focus sur les classes faibles} : Augmenter les échantillons des classes faibles
\item \textbf{Apprentissage par transfert} : Utilisation de modèles pré-entraînés
\end{itemize}

\subsection{Impact Attendu}

L'atteinte de ces objectifs positionnera notre système parmi les solutions les plus performantes du marché, avec une fiabilité adaptée aux applications critiques nécessitant une précision extrême.

\section{GLOSSAIRE DES MÉTRIQUES}

\subsection{Exactitude (Accuracy)}

\begin{equation}
\text{Exactitude} = \frac{\text{Nombre de prédictions correctes}}{\text{Nombre total de prédictions}}
\end{equation}

\textbf{Exemple} : 96 prédictions correctes sur 100 → Exactitude = 96\%

\subsection{Précision (Precision)}

\begin{equation}
\text{Précision} = \frac{\text{Vrais Positifs}}{\text{Vrais Positifs + Faux Positifs}}
\end{equation}

\textbf{Exemple} : 85 véritables "A" parmi 90 prédictions "A" → Précision = 94.4\%

\subsection{Rappel (Recall)}

\begin{equation}
\text{Rappel} = \frac{\text{Vrais Positifs}}{\text{Vrais Positifs + Faux Négatifs}}
\end{equation}

\textbf{Exemple} : 92 "A" correctement identifiés sur 100 véritables "A" → Rappel = 92\%

\subsection{Score F1}

\begin{equation}
\text{F1} = 2 \times \frac{\text{Précision} \times \text{Rappel}}{\text{Précision} + \text{Rappel}}
\end{equation}

\textbf{Exemple} : Précision = 90\%, Rappel = 95\% → F1 = 92.4\%

\subsection{mAP (Mean Average Precision)}

Moyenne des précisions sur différentes classes et seuils de confiance. Le mAP50 considère un chevauchement de 50\% entre les boîtes détectées et les vérités terrain.

\subsection{CrossEntropyLoss}

Mesure l'erreur entre les distributions de probabilités prédites et réelles :

\begin{equation}
\mathcal{L}_{\text{CE}} = -\sum_{i=1}^{N} y_i \log(\hat{y}_i)
\end{equation}

où $y_i$ est la vérité terrain et $\hat{y}_i$ la prédiction.

\section{CONCLUSIONS}

Notre recherche démontre la supériorité de l'architecture hybride pour la reconnaissance de plaques d'immatriculation. La séparation des tâches de détection et classification permet non seulement d'améliorer les performances métriques, mais aussi de réduire significativement les coûts de développement grâce à l'annotation simplifiée.

L'approche hybride atteint une exactitude de 95.85\% avec une réduction de 75\% du temps d'annotation, tout en maintenant une infrastructure adaptable et évolutive. Les perspectives d'optimisation vers un F1-score de 99\% et une CrossEntropyLoss de $1 \times 10^{-3}$ positionnent cette solution comme compétitive pour des applications industrielles exigeantes.

\addtolength{\textheight}{-12cm}   % This command serves to balance the column lengths
                                  % on the last page of the document manually. It shortens
                                  % the textheight of the last page by a suitable amount.
                                  % This command does not take effect until the next page
                                  % so it should come on the page before the last. Make
                                  % sure that you do not shorten the textheight too much.

%%%%%%%%%%%%%%%%%%%%%%%%%%%%%%%%%%%%%%%%%%%%%%%%%%%%%%%%%%%%%%%%%%%%%%%%%%%%%%%%

\end{document}
