%%%%%%%%%%%%%%%%%%%%%%%%%%%%%%%%%%%%%%%%%
% Présentation Beamer - ALPR
% Reconnaissance Automatique des Plaques d'Immatriculation
%%%%%%%%%%%%%%%%%%%%%%%%%%%%%%%%%%%%%%%%%

\documentclass[
	11pt,
	aspectratio=169,
]{beamer}

\graphicspath{{Images/}{./}}

\usepackage{booktabs}
\usepackage{amsmath}
\usepackage[utf8]{inputenc}
\usepackage[T1]{fontenc}

%----------------------------------------------------------------------------------------
%	THÈME ET COULEURS
%----------------------------------------------------------------------------------------

\usetheme{Madrid}
\useinnertheme{circles}
\usepackage[default]{opensans}

%----------------------------------------------------------------------------------------
%	INFORMATIONS DE LA PRÉSENTATION
%----------------------------------------------------------------------------------------

\title[ALPR]{Rapport 2025-10-28 : Système de Reconnaissance Automatique des Plaques d'Immatriculation} % The short title in the optional parameter appears at the bottom of every slide, the full title in the main parameter is only on the title page
\subtitle{Du Monolithique à l'Hybride : Une Évolution Stratégique}

\author[Équipe IA]{Équipe de Recherche en Intelligence Artificielle}

\institute[UATM]{
	UATM GASA FORMATION\\
	\smallskip
	\textit{info@uatm-gasa.com}
}

\date[28 Oct 2025]{28 Octobre 2025}

%----------------------------------------------------------------------------------------

\begin{document}

%----------------------------------------------------------------------------------------
%	PAGE DE TITRE
%----------------------------------------------------------------------------------------

\begin{frame}
	\titlepage
\end{frame}

%----------------------------------------------------------------------------------------
%	TABLE DES MATIÈRES
%----------------------------------------------------------------------------------------

\begin{frame}
	\frametitle{Plan de la Présentation}
	\tableofcontents
\end{frame}

%----------------------------------------------------------------------------------------
%	INTRODUCTION
%----------------------------------------------------------------------------------------

\section{Introduction}

\begin{frame}
	\frametitle{Contexte et Enjeux}
	
	\begin{block}{Problématique}
		La lecture des numéro de plaque d'immatriculation pose des défis majeurs :
		\begin{itemize}
			\item Variabilité des conditions d'éclairage
			\item Diversité des angles de vue et des polices d'écriture
			\item Résolutions d'image variables liée à celle de la caméra
			de prise de vue
			\item Formats de plaques différents
		\end{itemize}
	\end{block}
	
	% \bigskip
	
	\begin{exampleblock}{Notre Objectif}
		Développer un système performant optimisant le compromis entre précision, coût et déploiement.
	\end{exampleblock}
\end{frame}

%----------------------------------------------------------------------------------------
%	APPROCHE MONOLITHIQUE
%----------------------------------------------------------------------------------------

\section{Le Passé : Approche Monolithique}

\begin{frame}
	\frametitle{Architecture Monolithique YOLOv11}

	\begin{columns}[c]
		\begin{column}{0.5\textwidth}
			\textbf{Caractéristiques principales :}
			\begin{itemize}
				\item Architecture unifiée
				\item Détection + classification simultanées
				\item Pipeline simple
				\item Traitement temps réel
			\end{itemize}
		\end{column}
		
		\begin{column}{0.5\textwidth}
			\textbf{Formule de complexité :}
			\begin{equation*}
				T_{annotation} = n \times t_{caracteres} \times c_{complex}
			\end{equation*}
			
			\bigskip
			
			\alert{Limitation majeure :} Annotation très coûteuse en temps et ressources humaines
			afin d'atteindre de bonne précision dans la classification des caractères.
		\end{column}
	\end{columns}
\end{frame}

%------------------------------------------------

\begin{frame}
	\frametitle{Performances du Modèle Monolithique}
	
	\begin{table}
		\centering
		\begin{tabular}{lc}
			\toprule
			\textbf{Métrique} & \textbf{Valeur} \\
			\midrule
			mAP50 & 96.87\% \\
			mAP50-95 & 69.20\% \\
			Précision & 94.97\% \\
			Rappel & 92.23\% \\
			\textbf{Score F1} & \textbf{93.58\%} \\
			Temps d'inférence & 333 ms \\
			Précision (conf. > 92\%) & 83.54\% \\
			\bottomrule
		\end{tabular}
	\end{table}
	
	\bigskip
	
	\begin{alertblock}{Problèmes identifiés}
		\begin{itemize}
			\item mAP50-95 modeste (69.20\%)
			\item Optimisation conflictuelle détection/classification
			\item Annotation très coûteuse (plusieurs minutes par image)
		\end{itemize}
	\end{alertblock}
\end{frame}

%----------------------------------------------------------------------------------------
%	ARCHITECTURE HYBRIDE
%----------------------------------------------------------------------------------------

\section{Le Présent : Architecture Hybride}

\begin{frame}
	\frametitle{Nouvelle Architecture : Approche Hybride}
	
	\begin{block}{Conception en Deux Modules}
		\begin{enumerate}
			\item \textbf{Module de détection} : YOLO optimisé pour localiser les caractères
			\item \textbf{Module de classification} : Réseau neuronal dédié à l'identification
		\end{enumerate}
	\end{block}
	
	\bigskip
	
	\begin{exampleblock}{Gain d'annotation}
		\begin{equation*}
			\text{Gain} = \frac{t_{monolithique} - t_{hybride}}{t_{monolithique}} = \mathbf{25\%}
		\end{equation*}
	\end{exampleblock}
	
	\bigskip
	
	\textbf{Annotation plus rapide !}
\end{frame}

%------------------------------------------------

\begin{frame}
	\frametitle{Performances de l'Architecture Hybride}
	
	\begin{columns}[c]
		\begin{column}{0.5\textwidth}
			\begin{table}
				\centering
				\begin{tabular}{lc}
					\toprule
					\textbf{Métrique} & \textbf{Valeur} \\
					\midrule
					mAP50 & 99.47\% \\
					mAP50-95 & 74.01\% \\
					Précision & 97.25\% \\
					Rappel & 95.85\% \\
					\textbf{Score F1} & \textbf{97.06\%} \\
					Inférence & 587 ms \\
					Conf. > 92\% & 99.21\% \\
					\bottomrule
				\end{tabular}
			\end{table}
		\end{column}
		
		\begin{column}{0.5\textwidth}
			\textbf{Amélioration notable :}
			\begin{itemize}
				\item \alert{+3.48\%} sur le F1-score
				\item \alert{+2.60\%} sur mAP50
				\item \alert{+15.67\%} sur précision haute confiance
			\end{itemize}
			
			\bigskip
			
			\textbf{Classes excellentes :}\\
			F1 = 0.99 pour 2, 4, 9, C, E, K, M
		\end{column}
	\end{columns}
\end{frame}

%------------------------------------------------

\begin{frame}
	\frametitle{Comparaison des Deux Approches}
	
	\begin{table}
		\centering
		\begin{tabular}{lcc}
			\toprule
			\textbf{Critère} & \textbf{Monolithique} & \textbf{Hybride} \\
			\midrule
			Score F1 & 93.58\% & \textcolor{green}{\textbf{97.06\%}} \\
			mAP50 & 96.87\% & \textcolor{green}{\textbf{99.47\%}} \\
			Précision (conf. > 92\%) & 83.54\% & \textcolor{green}{\textbf{99.21\%}} \\
			Temps d'inférence & \textcolor{green}{\textbf{333 ms}} & 587 ms \\
			Coût d'annotation & Élevé & \textcolor{green}{\textbf{-25\%}} \\
			Flexibilité & Limitée & \textcolor{green}{\textbf{Élevée}} \\
			\bottomrule
		\end{tabular}
	\end{table}
	
	\bigskip
	
	\begin{block}{Compromis}
		Gain substantiel en précision contre augmentation du temps de calcul
	\end{block}
\end{frame}

%------------------------------------------------

\begin{frame}
	\frametitle{Avantages de l'Architecture Hybride}
	
	\begin{columns}[t]
		\begin{column}{0.5\textwidth}
			\textbf{Avantages techniques :}
			\begin{itemize}
				\item Spécialisation des modules
				\item Optimisation indépendante
				\item Gestion améliorée des variations
				(police de caractères, couleur des caractères, etc...)
			\end{itemize}
		\end{column}
		
		\begin{column}{0.5\textwidth}
			\textbf{Avantages opérationnels :}
			\begin{itemize}
				\item Réduction de 25\% du temps d'annotation, donc plus rapide
				\item Infrastructure évolutive
				\item Maintenance facilitée
			\end{itemize}
		\end{column}
	\end{columns}
	
	\bigskip
	
	\begin{exampleblock}{Bilan}
		L'approche hybride offre un meilleur équilibre performance/coût avec une architecture adaptable.
	\end{exampleblock}
\end{frame}

%----------------------------------------------------------------------------------------
%	PERSPECTIVES
%----------------------------------------------------------------------------------------

\section{Le Future : Perspectives d'Amélioration}

\begin{frame}
	\frametitle{Objectifs Futurs}
	
	\begin{block}{Cibles Quantifiées}
		\begin{equation*}
			\text{F1-score}_{\text{cible}} = \mathbf{99\%}
		\end{equation*}
		\begin{equation*}
			\mathcal{L}_{\text{CrossEntropy}} = \mathbf{1 \times 10^{-3}}
		\end{equation*}
	\end{block}
	
	\bigskip
	
	\begin{columns}[t]
		\begin{column}{0.5\textwidth}
			\textbf{Stratégies données :}
			\begin{itemize}
				\item Collecte massive d'échantillons
				\item Data augmentation avancée
				\item Focus sur classes faibles
			\end{itemize}
		\end{column}
		
		\begin{column}{0.5\textwidth}
			\textbf{Stratégies modèles :}
			\begin{itemize}
				\item Fine-tuning architecture
				\item Apprentissage par transfert
				\item Optimisation du temps de calcul
			\end{itemize}
		\end{column}
	\end{columns}
\end{frame}

%------------------------------------------------

\begin{frame}
	\frametitle{Feuille de Route Stratégique}
	
	\begin{enumerate}
		\item \textbf{Court terme (1-3 mois)}
		\begin{itemize}
			\item Augmentation du dataset
			\item Amélioration des classes faibles (O, J, P, D)
		\end{itemize}
		
		\bigskip
		
		\item \textbf{Moyen terme (3-6 mois)}
		\begin{itemize}
			\item Optimisation architecturale
			\item Réduction temps de calcul (objectif : 50 FPS)
		\end{itemize}
		
		\bigskip
		
		\item \textbf{Long terme (6-12 mois)}
		\begin{itemize}
			\item Atteinte F1-score = 99\%
			\item Déploiement industriel
		\end{itemize}
	\end{enumerate}
\end{frame}

%----------------------------------------------------------------------------------------
%	MÉTRIQUES
%----------------------------------------------------------------------------------------

\section{Glossaire des Métriques}

\begin{frame}
	\frametitle{Métriques Clés : Définitions}
	
	\begin{block}{Précision}
		\begin{equation*}
			\text{Précision} = \frac{\text{Vrais Positifs}}{\text{Vrais Positifs + Faux Positifs}}
		\end{equation*}
		Parmi les prédictions positives, quelle proportion est correcte ?
	\end{block}
	
	\begin{block}{Rappel}
		\begin{equation*}
			\text{Rappel} = \frac{\text{Vrais Positifs}}{\text{Vrais Positifs + Faux Négatifs}}
		\end{equation*}
		Parmi les vrais positifs, quelle proportion est correctement prédicte ?
	\end{block}
\end{frame}

%------------------------------------------------

\begin{frame}
	\frametitle{Score F1 et Cross-Entropy}
	
	\begin{block}{Score F1}
		Moyenne harmonique entre précision et rappel :
		\begin{equation*}
			\text{F1} = 2 \times \frac{\text{Précision} \times \text{Rappel}}{\text{Précision} + \text{Rappel}}
		\end{equation*}
		\textbf{Exemple :} Précision = 90\%, Rappel = 95\% $\rightarrow$ F1 = 92.4\%
	\end{block}
	
	\begin{block}{Cross-Entropy Loss}
		Mesure l'erreur entre distributions prédites et réelles :
		\begin{equation*}
			\mathcal{L}_{\text{CE}} = -\sum_{i=1}^{N} y_i \log(\hat{y}_i)
		\end{equation*}
		Plus cette valeur est faible, meilleure est la confiance du modèle.
	\end{block}
	\bigskip
\end{frame}

%----------------------------------------------------------------------------------------
%	CONCLUSIONS
%----------------------------------------------------------------------------------------

\section{Conclusions}

\begin{frame}
	\frametitle{Conclusions et Perspectives}
	
	\begin{block}{Résultats Obtenus}
		\begin{itemize}
			\item Architecture hybride \alert{supérieure} à l'approche monolithique
			\item F1-score de \textbf{97.06\%} (+3.48\%)
			\item Réduction de \textbf{25\%} du temps d'annotation
			\item Infrastructure évolutive et maintenable
		\end{itemize}
	\end{block}
	
	\bigskip
	
	\begin{exampleblock}{Vision Future}
		Objectif F1 = 99\% positionne cette solution comme
		\textbf{compétitive pour des applications industrielles exigeantes}.
	\end{exampleblock}
	
	\bigskip
	
	\begin{alertblock}{Prochaines Étapes}
		Focus sur l'optimisation des classes faibles et la réduction du temps de calcul.
	\end{alertblock}
\end{frame}

%----------------------------------------------------------------------------------------
%	REMERCIEMENTS
%----------------------------------------------------------------------------------------

\begin{frame}
	\frametitle{Remerciements}
	
	% \begin{columns}[t]
	% 	\begin{column}{0.45\textwidth}
	% 		\textbf{Équipe de Recherche}
	% 		\begin{itemize}
	% 			\item Architectures Deep Learning
	% 			\item Traitement d'images
	% 			\item Évaluation performances
	% 		\end{itemize}
	% 	\end{column}
		
	% 	\begin{column}{0.5\textwidth}
	% 		\textbf{Institution}
	% 		\begin{itemize}
	% 			\item UATM GASA FORMATION
	% 			\item Laboratoire de Recherche en IA
	% 		\end{itemize}
	% 	\end{column}
	% \end{columns}
	
	% \bigskip\bigskip
	
	\centering
	\textbf{Merci pour votre attention !}
\end{frame}

%----------------------------------------------------------------------------------------
%	QUESTIONS
%----------------------------------------------------------------------------------------

\begin{frame}[plain]
	\begin{center}
		{\Huge Questions ?}
		
		\bigskip\bigskip
		
		{\LARGE Commentaires et discussions}
		
		\bigskip\bigskip
		
		\textit{info@uatm-gasa.com}
	\end{center}
\end{frame}

%----------------------------------------------------------------------------------------

\end{document}
